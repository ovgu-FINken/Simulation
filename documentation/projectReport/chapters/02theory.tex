\chapter{Theory}
\label{sec:theo}

%In diesem Kapitel beschreiben Sie Ihren eigenen Beitrag
%- Es muss klar sein, worin die eigentliche Innovation besteht#

\section{Quadcopter Modelling}

\todo{fundamental physics}

\todo{particle simulation}



\section{Vrep}

\todo{connecting visual representation and physical model}

\todo{simulation structure (lua scripts, scene structure}

\todo{lua module structure}

\todo{external interface (signals)}

\section{Communication V-REP-Quadrocopter}
\textbf{\textit{Goal}:} our mixed reality simulation needs a dependable link of communication between the V-REP simulation environment and the flying quadrocopters. The Quadcopter needs to stream its telemetry data in real-time to the V-REP, and the reverse communication is needed as well.
The communication between the V-REP and the quadrocopters passes through several software components, which are depicted on figure \ref{fig:communication}. 

\begin{figure}[h!]
 \begin{center}
  \includegraphics[scale=0.15]{communication.png}
 \end{center}
  \caption{communication V-REP - Quadrocopter\label{fig:communication}}
\end{figure}

\textbf{\textit{V-REP}} provides several means of communication with an external application. One of them is the Remote API, which allows to control a simulation (or the simulator itself) from an external application or a remote hardware (e.g. real robot, remote computer, etc.). The V-REP remote API is composed by approximately one hundred functions that can be called from a C/C++ application, a Python script, a Java application, a Matlab/Octave program, an Urbi script, or a Lua script. The remote API functions are interacting with V-REP via socket communication in a way that reduces lag and network load to a great extent.

\textbf{\textit{Java API}} is the external program, that communicates with V-REP through the Remote API.
We have chosen to implement our external program, communicating with the V-REP, in the Java programming language regarding the following advantages: Java's platform independence allows to run the external program even on different machine with different operating system than the one used for running the V-REP environment.  Java is object-orientated which favours the use of design patterns and highly abstraction layers, which allows us to write an API that is modular, reusable and can later be easily extended to support other mixed-reality scenarious. The implementation and architecture of the Java API is duscussed in details in \ref{sec:implementation}. The purpose of the Java application is to serve as a communicating bridge between the Paparazzi Ground Station and the V-REP. It detects all quadrocopters in the V-REP simulation, builds their virtual representations and feeds the models with real-time data.

\textbf{\textit{Ivy Bus}} is a simple protocol and a set of open-source (LGPL) libraries and programs that allows applications to broadcast information through text messages, with a subscription mechanism based on regular expressions. Ivy libraries are available in C, C++, Java, Python and Perl, on Windows and Unix boxes and on Macs. The Paparazzi Ground Station uses the Ivy Bus as a means of communication between the different software components. Figure \ref{fig:paparazziGS} depicts the communication structure in the Paparazzi Ground Station, in which the different agents communicate with each other by sending messages on the Ivy-Bus.

\begin{figure}[h!]
 \begin{center}
  \includegraphics[scale=0.7]{paparazzi_gs.png}
 \end{center}
  \caption{Ivy-Bus in Paparazzi Ground Station\label{fig:paparazziGS}}
\end{figure}


% example for definition    
% \begin{definitionnonum}[Softwarearchitektur ]
% Die Softwarearchitektur repräsentiert alle Softwarekomponenten und deren Interaktionen in einer hierarchischen Struktur. Es werden sowohl statische Aspekte wie Schnittstellen und Datenpfade zwischen Softwarekomponenten, als auch dynamische Aspekte wie Prozessabläufe und zeitliches Verhalten beschrieben.
% \end{definitionnonum} 
 

  
%  example for bulletpoints
%\begin{itemize}    
%	\item{(überarbeiteter) Sicherheitsplan nach ISO 26262-6:2011, 5.5.1}
%	\item{Design- und Programmierrichtlinien für Programmier- und Modellierungssprachen nach ISO 26262-6:2011, 5.5.5}
%	\item{Hardware-Software-Interfacespezifikation nach ISO 26262-4:2011, 7.5.6}
%	\item{Software-Sicherheitsanforderungen nach ISO 26262-6:2011, 6.5.1}
%	\item{(überarbeiteter) Software-Verifizierungsplan nach  ISO 26262-6:2011, 6.5.3}
%	\item{Software-Verifizierungsbericht nach ISO 26262-6:2011, 6.5.4}
%\end{itemize}

 
%example for table    
% \begin{table}[h]
%      \centering
%    \caption{7.4.1 Notationen für Softwarearchitekturen\cite{iso6}}
%    \begin{tabular}{|c|l|c|c|c|c|}
%      \hline
%     \multicolumn{2}{|c|}{\multirow{2}{*}{Methoden}} & \multicolumn{4}{|c|}{ASIL}\\
%      \multicolumn{2}{|c|}{} &A & B & C & D\\
%      \hline
%       1a & Informelle Notationen & ++ & ++ & + & +\\
%      \hline
%       1b & Semi-formale Notationen & + & ++ & ++ & ++\\
%      \hline
%        1c & Formale Notationen & + & + & + & +\\
% 
%      \hline
%      \end{tabular}
%      \label{tab:archDescr}
%\end{table}
