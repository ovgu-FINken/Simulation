\ifgerman{\chapter{Einleitung}}{\chapter{Introduction}}
%- Hintergrund
%- Motivation
%- Ziele
%- Aufgaben
%- Allgemeine Beschreibung des Projektes
%- Worum geht es in dieser Arbeit?
%- Wer hat die Arbeit veranlasst und wozu?
%- Wer soll von den Ergebnissen profitieren?
%- Welches Problem soll gelöst werden? Warum?
%- Unter welchen Umständen braucht man eine Verbesserung?
%- Was ist der Stand der Technik?
%- Welche noch offenen Probleme gibt es?
%- Worin unterscheidet sich mein Ansatz von den bisherigen?
%- Welche Ziele hat die Arbeit?
%- Wie will ich diese Ziele erreichen?
%- Was habe ich im Einzelnen vor?


\section{Motivation}

The work presented in this document was carried out at the Swarm Lab at the Otto-von-Guericke-University Magdeburg. The research focus of the working group lies on implementing and investing swarm algorithms in practice by using small indoor quadcopter. The used FINken quadcopter were developed in association with the working group and are small, but powerfull and are highly extensible. As the research focus on swarm intelligence puts an interest on autonomous behaviour, the copter fly without any external reference and rely solely on onboard sensors.

As the copter were developed in the working group and were designed with a focus on modularity, they are under constant change. Testing new changes of the copter, e.g. new control or behavioural algorithms alway poses a certain risk to the hardware due to crashs caused by bugs. Thus, a simulation tool for the quadcopters to test new software is desirable to be able to do a safe first evaluation of newly implemented ideas. A less abstract solution than a pure simulation could be a mixed reality simulation, where the behaviour of real and simulated quadcopter could be directly compared.

\todo{who needs the results}

\todo{what are the problems to be solved?}

\todo{what are existing solutions, what's different in this approach, what is the improvement}


       


  
\section{Problem Statement}
    
    \todo{what are the goals}
    \todo{how are we going to reach this goals}
    \todo{what is to be done}
 
         
\section{Outline}
    
  \todo{short description of the sections}
   % use " in \ref{} to reference "\labels{}" in the document
   