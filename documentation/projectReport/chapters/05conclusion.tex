\chapter{Conclusion}
\label{sec:conclusion}


\section{}
\todo{do the results show that it works?}



\section{Future Work}
%Die Beurteilung ist einer der wichtigsten Abschnitte der Arbeit
%- Sie enthält die Quintessenz des gesamten Projektes
%Viele lesen nur die Einführung und die Beurteilung an
%- Hier muss also alles Wichtige drin stehen!
%Hier beweisen Sie dass Sie …
%- die Aufgabe und deren Bedeutung verstanden haben
%- die Ergebnisse richtig zu interpretieren vermögen
%- wissen, worauf es bei diese Arbeit ankam




% example for definition
%      \begin{definitionnonum} [Hierarchische Struktur] 
%      Wenn eine Softwarestruktur als Menge von Komponenten  und Relationen zwischen den Komponenten beschrieben wird, gilt diese Struktur als hierarchisch, wenn eine Relation oder ein Prädikat $R(\alpha, \beta)$ über Paaren dieser Komponenten Schichten folgendermaßen beschreiben kann: 
%      \begin{itemize} 
%      \item{Schicht 0 ist eine Menge von Komponenten $\alpha$, sodass  kein $\beta$ mit $R(\alpha,\beta)$ existiert} 
%      \item{Schicht $i$ ist eine Menge von Komponenten $\alpha$ ist, sodass auf Level $i-1$ ein $\beta$ existiert mit $R(\alpha,\beta)$ und alle $\gamma$ für $R(\alpha,\gamma)$ in der Schicht $i-1$ oder niedriger liegen} 
%      \end{itemize}
%      \end{definitionnonum}
      


% example for drawing      
%              \tikzstyle{decision} = [diamond, draw, fill=blue!20,
%    text width=4.5em, text badly centered, node distance=2.5cm, inner sep=0pt]
%\tikzstyle{block} = [rectangle, draw, fill=blue!20,
%    text width=5em, text centered, rounded corners, minimum height=4em]
%\tikzstyle{line} = [draw, very thick, color=black!50, -latex']
%\tikzstyle{cloud} = [draw, ellipse,fill=red!20, node distance=2.5cm,
%    minimum height=2em]
%\tikzstyle{node} = [draw, ellipse, node distance=2.5cm,
%    minimum height=2em]
%\tikzstyle{fullnode}= [draw, ellipse,fill=black!20, node distance=2.5cm,
%    minimum height =2em]
%\begin{figure}[h]
%\begin{center}
%\begin{tikzpicture}[scale=2, node distance = 2cm, auto]
%    % Place nodes
%  \node[node](b1){$b_1$};
%   \node[node, below of=b1, right of=b1](b3){$b_3$};
% \node[node, right of=b3, above of=b3](b2){$b_2$};
%
%
%\path [->, densely dotted] (b1) edge node{e} (b2);
%\path [->] (b1) edge (b3);
%\path [->] (b3) edge (b2);
%\path [->, dashed] (b1) edge[bend right] node[left]{$e_1$}(b3);
%\path [->,dashed] (b3) edge[bend right]  node[right]{$e_2$} (b2);
%\end{tikzpicture}
%\caption{Beispiel Gesetz von Demeter}
%\end{center}
%\end{figure}




% example for equation   
%\begin{equation}
%   L_{GvD}(a) = \sum_{b_i \in B(a)} L_{GvD}(b_i)
%\end{equation}
%    
      
